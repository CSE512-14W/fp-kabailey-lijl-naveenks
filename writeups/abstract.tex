\abstract{This project addresses the massive amount of data available to instructors of 
MOOCs (massive open online courses). While some of the data is noise---students
who never intend to participate, or fail to submit the majority of assignments
---much of it is potentially valuable data on what methods and components in a
course are effective. It is, however, massive in quantity. We intend to use the
data readily available from Coursera to provide some exploratory visualizations
for a generic MOOC class, generally tracking attrition and success rates.

More specifically, we will enable professors, who upload their own data, to add
ress a variety of questions, including but not limited to the following:
(1) Comparing two (or more) instances of the same course: was change X in 
  assignment 3 effective? Were the overall statistics comparable?
(2) Tracking characteristics throughout the course, based on intro 
  demographic information. Do people who don't know recursion do significantly
  worse on this quiz?
(3)racking the timeline of the course: when do people drop out? Can we 
  tell why? 
For the purposes of this project, we will be prototyping from Dan Grossman's 
data, and working with him to determine desirable visualizations.
}