Though we feel this project represents a solid prototype on the available data, there is a significant amount of work yet to be done in this area. We describe some of the improvements and extensions we would like to make below.

\paragraph{A Wealth of Information}
The amount of data available from the Coursera platform is mind-boggling. Though we used a simple set of grades, timestamps, and demographic data to be able to create this prototype, there is additional data that could provide more insight. Among other things, the use of forum participation, video viewings, and the optional in-video quizzes would provide a rich set of information from which to gauge engagement (frequency of participation) and understanding (frequency of questions and re-watched videos). Other pieces of data such as clicks, and total time to take an exam would be harder to gather, but perhaps more enlightening. 

In addition to supplementing our data with additional information from a single offering or course, we would like to be able to add data from across multiple offerings or multiple courses. With appropriate learning algorithms, we might begin to answer the questions of whether exams are actually helpful, if certain length videos or courses encourage retention and so on. MOOCs are, after all, a large source of rich data to compute over, and Big Data mechanisms applied would provide insights into the utility of certain components. 

\paragraph{Automation and Accessibility}
In addition to providing more data and insights to instructors, we would like to be able to provide easier access. Ideally, we would integrate our tool directly into platforms such as Coursera, though doing so would be a negotiation between development teams. If not that, than at least having a generic auto-upload feature that allows an instructor to upload .csv files and designate columns as demographic, assignment, or exam data would be useful in making the tool more widely available to instructors. As is, the tool is a bit too brittle to enable such usage. 
 
\paragraph{Alternate Viewpoints}
Last of our long-term goals is to look at data sources other than MOOCs. Both students and professors had many suggestions for other uses of the data and resultant visualizations, both for classical classes in a classroom, and for exposing the difficult portions of a course. Exploring these options would provide richer platform. 