Often, when course instructors seek information about their in-progress or completed courses, they 
turn to direct data manipulation in software such as Microsoft Excel, Numbers, or Google Docs \cite{ excel, numbers, docs}. 
However, with expansion of educational institutions into course styles such as Massive Open Online 
Courses (MOOCs), the straightforward manipulation of data to find outliers and students who need 
additional assistance becomes too overwhelming for the individual professor to manage in a 
reasonable time frame. 

With courses that enroll tens of thousands of participants in a single offering of a single 
course, it is easy for instructors to be overwhelmed by the sheer amount of data gathered by such 
courses, both in terms of demographics as well as actual results on assignments. Sites such as 
Coursera \cite{coursers} which facilitate the MOOC offerings through interfaces for posting videos as
well as posting, submitting, and grading assignments provide limited interfaces for following a single student---or even a group of students---over the weeks of a course. Nor do they provide any satisfying visualization of the impact of the course elements, despite what is undoubtedly a surfeit of data. 

This project comes in response to the questions of one such professor, who wanted to answer several questions about the programming languages class he had offered twice in the past. Starting with an exploration of the current Coursera data available and their visualizations, moving to working with data (modified for anonymization) generated by actual students, we sought to provide better visualization of the data with several main goals. 

\paragraph{Accessibility}
By far the largest problem with the data in its initial form was accessibility and tractability for 
instructors: though they could get the data into .csv format, it was largely impractical to do calculations
over the data without significant number churning. We provide visualizations with clustering and 
trends so that the data and emerging patterns can be more immediately accessible to the instructors.

\paragraph{Clustering and Comparison}
In addition to visualizing the data in entirety, we allow the instructors to select groups of students, 
through course offering, demographic information, or percentile groups, such that the trends between 
two different student groups can easily be visualized. Providing aggregate data over these groups removes the noise inherent in large datasets, and isolates the important trends. 

\paragraph{Component Impact}
The last important feature we offer to instructors is the ability to isolate the impact of specific components of the course. Whether through the drop-off rate visualized on a time line or the exam questions viewed as a progression through course material, or a best-predictor type view for the final grades, we want instructors to determine which components of their course have the largest and most accurate impact on student retention and success rate (as represented through grades). 

With these goal in mind, we create a visual exploration tool for exploring MOOCs and their data. We 
go over the background and previous work in the next section, explaining the design and methodology
in the following section, and concluding with results and feedback that lead to future work. 

{\bf A note on student privacy. } One of the most delicate aspects of a tool like this is that student data is inherently private. Though one professor did generously give us access to the data from two offerings of his course, much of the demographic data, for example, is not easily available from Coursera due to privacy concerns. Further, tracking students' performance of time may be identifiable in and of itself. For the purposes of this demonstration and writeup we have used the provided data that has been fully anonymized in terms of demographics and slightly tweaked, by the numbers, to ensure further privacy. We provide additional explanations of this in the methodology section.