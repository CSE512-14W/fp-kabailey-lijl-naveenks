
\subsection{Coursera}

\subsection{D3}
D3.js is a javascript library for manipulating and visualizing data. It takes advantages of SVG, javascript, HTML5 and CSS to provide easy to use and powerful ways to create dynamic visualizations. D3 simplifies data reading, parsing and transformation by providing special support for JSON, geoJSON, CSV and TSV files. It also contains a rich set of templates that greatly simplifies data visualizations, including bar charts, pie charts, scatter plots, etc.

\subsection{Related Research}
We were unable to find previous work looking at courses on the scale of MOOC offerings. 

Previous work seems isolated into a few different areas: analyzing student progress over years of standardized testing, and understanding the progress of students through a single course. The latter is the most similar to what we intend to provide, albeit with different approaches. For example, the work in [1] analyzes students and places them into states, creating a tree-like flow chart of where a student is likely to go next in the course. The overview of educational data mining (EDM) provided in [2] provides a very high-level description of where previous work has gone with respect to evaluating the effectiveness of courses, online material, advising, and so on. Much of this data is either not available to us or is outside the scope of our work, though the clustering techniques we implement are similar in nature to those discussed there. We intend to provide data specifically targeted at the world of MOOCs, where attrition is high, interaction with students is minimal, and the scale of data is too enormous to consider looking at some of the factors involved (like clicks on the course website). Moreover, we would like to provide an overview in a dashboard-like setting for the instructors of the course, to determine utility and effectiveness. 

Outside the scope of EDM, work has gone on in several areas with respect to standardized ?high stakes? exams (e.g. state tests through elementary and high school). Bendinelli and Marder [3] model the data a flow problem, and provide some basis for analysis based on trends and demographic characteristics. However, the data in that paper is severely restricted in scope, and provides minimal analysis at tracking a single individual. Other work looks at the usage data, but not performance [4], and still others are restricted only to looking at progress over several courses [5].
